% ==================================
%            Chapter 3.5
%       Quadratic Inequalities
%       Created by Michael Tang
%             2025.02.25
% ==================================

\subsection{Quadratic Inequalities}
\begin{itemize}
    \item \textbf{Key Concept:} A quadratic inequality involves a quadratic expression (e.g., $ax^2 + bx + c$) and a comparison to
    zero. The solution set is determined by finding where the quadratic expression is positive or negative.
    \item \textbf{Steps to Solve Quadratic Inequalities}
    \begin{itemize}
        \item Rearrange the inequality so that the quadratic expression is on one side and zero is on the other.
        \item Solve the corresponding quadratic equation to find the critical values (i.e., where the quadratic expression equals
        zero).
        \item Sketch the graph of the quadratic function and use the graph to determine the solution intervals.
    \end{itemize}
    \item \textbf{Set Notation for Quadratic Inequalities}
    \begin{itemize}
        \item If the quadratic expression is greater than zero (i.e., the curve is above the x-axis), the solution is the range
        of x-values where the curve lies above the x-axis. Example: $x < -3$ or $x > \frac{1}{2}$, written as
        $\{x : x < -3 \quad \text{or} \quad x > \frac{1}{2}\}$.
        \item If the quadratic expression is less than zero  (i.e., the curve is below the x-axis), the solution is the range of
        x-values where the curve lies below the x-axis. Example: $-3 < x < \frac{1}{2}$, written as
        $\{x : -3 < x < \frac{1}{2}\}$.
    \end{itemize}
    \item \textbf{General Tips for Quadratic Inequalities}
    \begin{itemize}
        \item \textbf{Critical Values:} These are the x-values where the quadratic expression equals zero. They are found by
        solving the corresponding quadratic equation.
        \item \textbf{Sketching the Graph:} Determine the direction of the parabola (upward or downward) based on the leading
        coefficient $a$.
        \item \textbf{Sign Analysis:} Use the sign of the quadratic expression in each interval defined by the critical values
        to determine where the inequality holds.
    \end{itemize}
\end{itemize}