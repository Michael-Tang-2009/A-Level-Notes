% ===============================================
%                 Chapter 1B.3
%             Displacement Reactions
%             Created by Michael Tang
%                  2025.01.01
% ===============================================

\subsubsection{1B.3 \underline{Displacement Reactions} (置换反应)}
\paragraph{What is a Displacement Reaction?} A displacement reaction occurs when a more reactive element replaces a less reactive
element in a compound. These reactions are often \underline{redox reactions} (氧化还原反应) where:
\begin{itemize}
    \item The more reactive element is \textbf{oxidized} (氧化) and loses electrons.
    \item The less reactive element is \textbf{reduced} (还原) and gains electrons.
\end{itemize}

\paragraph{Displacement Reactions Involving Metals}
\begin{itemize}
    \item[1.] General Characteristics:
    \begin{itemize}
        \item A metal reacts with the compound of aother metal.
        \item Produces a new metal and a new compound.
        \item Involves electron transfer (redox).
    \end{itemize}
    \item[2.] Examples:
    \begin{itemize}
        \item Reaction 1:
        \begin{equation}
            \ce{Mg(s) + CuSO4(aq) -> MgSO4(aq) + Cu(s)}
        \end{equation}
        \begin{itemize}
            \item \ce{Mg} oxidized to \ce{Mg^2+}.
            \item \ce{Cu^2+} reduced to \ce{Cu}.
        \end{itemize}
        \item Reaction 2 (Thermite Reaction 铝热反应):
        \begin{equation}
            \ce{Al(s) + Fe2O3(s) -> Al2O3(s) + Fe(s)}
        \end{equation}
        \begin{itemize}
            \item \ce{Al} oxidized to \ce{Al^3+}.
            \item \ce{Fe^3+} reduced to \ce{Fe}.
        \end{itemize}
    \end{itemize}
    \item[3.] Differences Between Reactions:
    \begin{itemize}
        \item Reaction 1: Occurs in aqueous solution.
        \item Reaction 2: Requires high temperatures; used in industrial processes like railway welding.
    \end{itemize}
    \item[4.] Ionic Equations\par
    Example: Reaction 1
    \begin{equation}
        \ce{Mg(s) + Cu^2+(aq) -> Mg^2+(aq) + Cu(s)}
    \end{equation}
\end{itemize}

\paragraph{Displacement Reactions Involving Halogens}
\begin{itemize}
    \item General Concept:
    \begin{itemize}
        \item A more reactive \underline{halogens} \footnote{Halogens are Group 7 elements in the periodic table, including
        fluorine (\ce{F}), chlorine (\ce{Cl}), bromine (\ce{Br}), iodine (\ce{I}), and astatine (\ce{At}).} (卤素) displaces a
        less reactive halogen from its compound.
        \item Follows the order of reactivity: \ce{F} $>$ \ce{Cl} $>$ \ce{Br} $>$ \ce{I}.
    \end{itemize}
    \item Example:
    \begin{itemize}
        \item Reaction of chlorine with potassium bromine. The full, ionic and simplifi ed ionic equations for this reaction are:
        \begin{equation}
            \begin{split}
                \ce{Cl2(aq) + 2KBr(aq) &-> 2KCl(aq) + Br2(aq)}\\
                \ce{Cl2(aq) + 2Br-(aq) + 2K+(aq) &-> 2K+(aq) + 2Cl-(aq) + Br2(aq)}\\
                \ce{Cl2(aq) + 2Br-(aq) &-> 2Cl-(aq) + Br2(aq)}
            \end{split}
        \end{equation}
    \end{itemize}
    \item Key Points: Halogen displacement reactions are redox reactions. Note that the reactivity of halogens decreases down
    the group.
\end{itemize}

\paragraph{Applications of Displacement Reactions} Industrial Welding:
\begin{itemize}
    \item The thermite reaction is used to join railway tracks.
    \item The exothermic reaction produces \underline{molten iron} (熔融铁), which fills the gap between tracks.
\end{itemize}