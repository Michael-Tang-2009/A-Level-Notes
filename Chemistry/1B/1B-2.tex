% ===============================================
%                 Chapter 1B.2
%  Techniques for Measuring the Rate of Reaction
%             Created by Michael Tang
%                  2024.12.31
% ===============================================

\subsubsection{1B.2 Typical Reactions of Acids}
\paragraph{Introduction} Acids are common reagents in chemistry, used to prepare salts. Examples of acid include
\underline{hydrochloric acid} (\ce{HCl} 氯化氢/盐酸), \underline{nitric acid} (\ce{HNO3} 硝酸),
\underline{sulfuric acid} (\ce{H2SO4} 硫酸), and \underline{phosphoric acid} (\ce{H3PO4} 磷酸).

\paragraph{Acid with Metals}
\begin{itemize}
    \item \textbf{General Equation:}
    \begin{equation}
        \ce{\text{Metal} + \text{Acid} -> \text{Salt} + \text{Hydrogen Gas}}
    \end{equation}
    \item \textbf{Example:}
    \begin{equation}
        \begin{split}
            \ce{Mg + 2HCl &-> MgCl2 + H2}\\
            \ce{Mg(s) + 2H+(aq) &-> Mg^{2+}(aq) + H2(g)}
        \end{split}
    \end{equation}
    \item \textbf{Explanation:} Hydrogen ions (\ce{H+}) are reduced to hydrogen gas (\ce{H2}), so these are
    \underline{redox reactions} (氧化还原反应), not \underline{neutralization reactions} (中和反应).
\end{itemize}

\paragraph{Acids with \underline{Metal Oxides} (金属氧化物) and Insoluble \underline{Metal Hydroxides} (金属氢氧化物)}
\begin{itemize}
    \item \textbf{General Equation:}
    \begin{itemize}
        \item Metal Oxide:
        \begin{equation}
            \ce{\text{Metal Oxide} + \text{Acid} -> \text{Salt} + \text{Water}}
        \end{equation}
        \item Metal Hydroxide:
        \begin{equation}
            \ce{\text{Metal Hydroxide} + \text{Acid} -> \text{Salt} + \text{Water}}
        \end{equation}
    \end{itemize}
    \item Typical equations for copper(II) oxide and zinc hydroxide reacting with sulfuric acid are:
    \begin{equation}
        \begin{split}
            \ce{CuO + H2SO4 &-> CuSO4 + H2O}\\
            \ce{CuO(s) + 2H+(aq) &-> Cu^{2+}(aq) + H2O(l)}\\
            \ce{Zn(OH)2 + H2SO4 &-> ZnSO4 + 2H2O}\\
            \ce{Zn(OH)2(s) + 2H+(aq) &-> Zn^{2+}(aq) + 2H2O(l)}
        \end{split}
    \end{equation}
    \item \textbf{Explanation:} These are \underline{neutralization reactions} (中和反应) with no change in
    \underline{oxidation numbers} \footnote{\textbf{Oxidation numbers:} An oxidation number is the charge that an atom would have
    if all bonds in a compound were ionic.\par
    \textbf{Rules for Assigning Oxidation Numbers}
    \begin{itemize}
        \item[1.] \textbf{Elements:} An atom in its elemental form (e.g., \ce{O2}, \ce{H2}, \ce{Na}) has an oxidation number of 0.
        \item[2.] \textbf{Simple Ions:} The oxidation number of a monatomic ion is equal to its charge (e.g., \ce{Na+}$=+1$,
        \ce{Cl-}$=-1$).
        \item[3.] \textbf{Compound:}
        \begin{itemize}
            \item The sum of oxidation numbers in a \underline{neutral compound} (中性化合物) is 0.
            \item In \underline{polyatomic ions} (多原子离子), the sum of oxidation numbers is equal to the ion's charge.
        \end{itemize}
        \item[4.] \textbf{Common Elements:}
        \begin{itemize}
            \item Group 1 elements (alkali metals 碱金属): Always $+1$.
            \item Group 2 elements (alkaline earth metals 碱土金属): Always $+2$.
            \item Hydrogen: $+1$ (except in peroxides $-1$, e.g., \ce{H2O2}, or when bonded to fluorine $+2$).
            \item Fluorine: Always $-1$.
        \end{itemize}
    \end{itemize}\par
    \textbf{Oxidation and Reduction}
    \begin{itemize}
        \item \textbf{Oxidation:} Increase in oxidation number.
        \item \textbf{Reduction:} Decrease in oxidation number.
        \item Example:
        \begin{equation}
            \ce{2Mg + O2 -> 2MgO}
        \end{equation}
        \item Magnesium (\ce{Mg}) changes from $0$ to $+2$ (oxidized).
        \item Oxygen (\ce{O2}) changes from $0$ to $-2$ (reduced).
    \end{itemize}} (氧化数).
\end{itemize}

\paragraph{Acids with \underline{Alkalis} (碱)}
\begin{itemize}
    \item \textbf{General Equation:}
    \begin{equation}
        \ce{\text{Acid} + \text{Alkali} -> \text{Salt} + \text{Water}}
    \end{equation}
    \item \textbf{Example:}
    \begin{equation}
        \begin{split}
            \ce{NaOH +H3PO4 &-> NaH2PO4 + H2O}\\
            \ce{2NaOH + H3PO4 &-> Na2HPO4 + 2H2O}\\
            \ce{3NaOH + H3PO4 &-> Na3PO4 + 3H2O}\\
            \ce{H+(aq) + OH-(aq) &-> H2O(l)}
        \end{split}
    \end{equation}
    \item \textbf{Explanation:} These are neutralization reactions with no oxidation state changes.
\end{itemize}

\paragraph{Acids with \underline{Carbonates} (碳酸盐)}
\begin{itemize}
    \item \textbf{General Equation:}
    \begin{equation}
        \ce{\text{Acid} + \text{Metal Carbonate} -> \text{Salt} + \text{Water} + \text{Carbon Dioxide}}
    \end{equation}
    \item \textbf{Example:}
    \begin{equation}
        \begin{split}
            \ce{Li2CO3 + 2HCl &-> 2LiCl + H2O + CO2}\\
            \ce{CO3_{2-}(aq) + 2H+(aq) &-> H2O(l) + CO2(g)}
        \end{split}
    \end{equation}
    \item \textbf{Explanation:} These are neutralization reactions with the release of carbon dioxide gas.
\end{itemize}

\paragraph{Acids with \underline{Hydrogencarbonates} (碳酸氢盐)}
\begin{itemize}
    \item \textbf{General Equation:}
    \begin{equation}
        \ce{\text{Acid} + \text{Metal Hydrogencarbonate} -> \text{Salt} + \text{Water} + \text{Carbon Dioxide}}
    \end{equation}
    \item \textbf{Example:}
    \begin{equation}
        \ce{NaHCO3 + HCl -> NaCl + H2O + CO2}
    \end{equation}
    \item \textbf{Application:} Used in baking to release carbon dioxide gas, making baked goods light and fluffy.
    \item \textbf{Test for Carbonates and Hydrogencarbonates:} Add an acid and test the gas produced with limewater
    (\ce{Ca(OH)2} 澄清石灰水).
\end{itemize}