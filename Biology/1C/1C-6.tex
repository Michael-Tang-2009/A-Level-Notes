% =========================================
%              Chapter 1C.6
%           Using the Evidence
%          Created by Michael Tang
%               2025.02.13
% =========================================

\subsubsection{1C.6 Using the Evidence}
\paragraph{Prevention is Better Than Cure}
\begin{itemize}
    \item CVDs cost individuals, families, and society a lot in terms of healthcare and lost productivity.
    \item Prevention is more cost-effective than treatment. Healthy habits can prevent the need for expensive medical
    \underline{interventions} (干预).
\end{itemize}

\paragraph{Overweight or Underweight}
\begin{itemize}
    \item Being overweight or underweight both increase the risk of CVDs.
    \item Regular physical activity helps protect against CVDs by improving cardiovascular fitness. The more oxygen used during
    exercise, the lower the CVD risk.
\end{itemize}

\paragraph{Smoking and CVD Risk}
\begin{itemize}
    \item Smoking is one of the leading risk factors for CVD. Smokers are at a higher risk of developing heart disease. Research
    shows that people who quit smoking after 1 year see a significant reduction in their heart disease risk.
    \item \textbf{Global Smoking Statistics:} In the UAE, 21.9\% of people are smokers, with higher rates among men (24.8\%) than
    women (4.2\%).
\end{itemize}

\paragraph{Obesity and CVDs}
\begin{itemize}
    \item Obesity is linked to a higher risk of developing CVDs due to increased fat storage, which can lead to health
    \underline{complications} (并发症) like \underline{hypertension} (高血压) and \underline{poor blood circulation} (血液循环不良).
    \item \textbf{Diet and Exercise:} Eating a balanced diet and exercising regularly can help maintain a healthy weight and
    reduce the risk of obesity and related heart diseases.
\end{itemize}

\paragraph{Salt and CVDs}
In many developed countries, excessive salt intake is linked to high blood pressure and CVDs. Reducing salt intake globally could
lower blood pressure and the incidence of CVDs.

\paragraph{Lifestyle Choices and Perception of Risk}
\begin{itemize}
    \item \textbf{Risk Perception:} People often miscalculate the risks of smoking, high-fat diets, and lack of exercise.
    Immediate rewards (e.g., pleasure from eating high-fat food) may be prioritized over long-term health risks.
    \item \textbf{Psychological Factors:} The habit of smoking or eating unhealthy food may be hard to break, even if people know
    the risks.
\end{itemize}

\paragraph{Global Public Health Efforts}
Governments and health organizations spend billions on campaigns to reduce smoking and promote healthy diets. However, some argue
this spending might be a waste if individuals don't change their behavior.