% ===============================
%          Chapter 1C.1
%  Risk, Correlation, and Cause
%     Created by Michael Tang
%           2025.01.13
% ===============================

\subsubsection{1C.1 Risk, Correlation, and Cause}
\paragraph{What is Risk?}
\begin{itemize}
    \item \textbf{Definition:}
    \begin{itemize}
        \item Risk is the probability of an event occurring.
        \item Example: The probability of picking a blue ball from a bag of six colored balls is 1 in 6, 0.1666, or 16.7\%.
    \end{itemize}
    \item \textbf{How Risk is Rerceived:}
    \begin{itemize}
        \item Perception is influenced by:
        \begin{itemize}
            \item Familiarity with the activity.
            \item Enjoyment of the activity.
            \item Approval or disapproval of the activity.
        \end{itemize}
        \item Actual mathematical risk may differ from personal perception.
        \item Example: Risk of dying in a car accident is 1 in 5747, but many people fear flying more than driving.
    \end{itemize}
\end{itemize}

\paragraph{\underline{Epidemiology} (流行病学) and Risk Factors}
\begin{itemize}
    \item \textbf{Epidemiology:}
    \begin{itemize}
        \item The study of disease patterns and their causes in populations.
        \item Identifies risk factors that increase the likelihood of disease.
    \end{itemize}
    \item \textbf{Risk Factors:}
    \begin{itemize}
        \item Factors that increase the probability of developing a diesase (e.g., smoking, obesity, and lack of exercise).
        \item Diseases like atherosclerosis are termed multifactorial disease as they result from multiple interacting factors.
    \end{itemize}
    \item \textbf{Correlation vs. Causation:}
    \begin{itemize}
        \item \textbf{\underline{Correlation} (相关性):} Two sets of data change together but do not imply cause and effect.
        \item \textbf{\underline{Causation} (因果关系):} A direct cause-effect relationship is established.
    \end{itemize}
\end{itemize}

\paragraph{Obesity and Cardiovascular Disease}
\begin{itemize}
    \item \textbf{Impact of Obesity:} Increased risk of diabetes and cardiovascular disease.
    \item \textbf{Why People Stay Obese:}
    \begin{itemize}
        \item Overestimation of benefits or underestimation of risks of unhealthy behaviors.
        \item Enjoyment of certain activities or foods.
        \item Lack of motivation or understanding of the risks.
    \end{itemize}
\end{itemize}