% ==================================
%            Chapter 3.7
%              Regions
%       Created by Michael Tang
%             2025.02.25
% ==================================

\subsection{Regions}
\begin{itemize}
    \item \textbf{Key Concept:} Regions on a graph can be represent the solution to inequalities. We can identify regions where
    the solution to an inequality holds by testing specific points and shading the appropriate region on the graph.
    \item \textbf{Shading Region for Inequalities}
    \begin{itemize}
        \item[1.] Shading for $y > f(x)$:
        \begin{itemize}
            \item The region where $y$ is greater than $f(x)$ is above the graph of $f(x)$.
            \item \textbf{Test Points:} Pick points in the region and check if they satisfy the inequality.
        \end{itemize}
        \item[2.] Shading for $y < f(x)$:
        \begin{itemize}
            \item The region where $y$ is less than $f(x)$ is below the graph of $f(x)$.
        \end{itemize}
    \end{itemize}
    \item \textbf{Graphical Representation}
    \begin{itemize}
        \item For linear inequalities, typically work with straight lines.
        \item For quadratic inequalities, use curves, and the shaded region represents the area where the inequality holds true.
    \end{itemize}
\end{itemize}