% ===============================================
%                 Chapter 1C.2
%          Calculations Involving Moles
%             Created by Michael Tang
%                  2025.01.04
% ===============================================

\subsubsection{1C.2 Calculations Involving Moles}
\paragraph{Definition of Mole}
\begin{itemize}
    \item A mole is the amount of substance that contains the same number of particles (atoms, molecules, ions, or electrons) as
    there are in 12 \unit{g} of carbon-12.
    \item The number is the Avogadro constant, $N_A = 6.022 \times 10^{23} \unit{mol^{-1}}$.
\end{itemize}

\paragraph{Counting Atoms}
\begin{itemize}
    \item Atoms are extremely small, so we use relative atomic masses to compare the masses of atoms.
    \item Example: Oxygen is 16 times heavier than hydrogen ($A_r$ of \ce{O} = 16.0, $A_r$ of \ce{H} = 1.0).
    \item A mole of oxygen atoms (16.0 \unit{g}) contains the same number of particles as a mole of hydrogen atoms (1.0 \unit{g}).
\end{itemize}

\paragraph{Calculations Using Moles}
\begin{itemize}
    \item \textbf{Equations for Calculating Moles}
    \begin{equation}
        \text{Amount in moles} (n) = \frac{\text{Mass in grams} (m)}{\text{Molar mass} (M)}
    \end{equation}
    Rearrangements:
    \begin{equation}
        m = n \times M \quad \text{or} \quad M = \frac{m}{n}
    \end{equation}
    \item \textbf{Worked Examples}
    \begin{itemize}
        \item \textbf{Substance in Moles:} What is the amount of soldium chloride in 6.15g ($M = 58.5$)?
        \begin{equation}
            n = \frac{m}{M} = \frac{6.51}{58.5} = 0.111 \unit{mol}
        \end{equation}
        \item \textbf{Mass of a Substance:} What is the mass of 0.263 mol of hydrogen iodide ($M = 127.9$)?
        \begin{equation}
            m = n \times M = 0.263 \times 127.9 = 33.6 \unit{g}
        \end{equation}
        \item \textbf{Molar Mass:} A sample has $n = 0.284$ mol and $m = 17.8$g. What is the molar mass?
        \begin{equation}
            M = \frac{m}{n} = \frac{17.8}{0.284} = 62.7 \unit{g \cdot mol^{-1}}
        \end{equation}
    \end{itemize}
\end{itemize}

\paragraph{What to Remember When Using Moles}
\begin{itemize}
    \item Always specify the type of particle (e.g., atoms, molecules, ions).
    \item For compounds, specify the formula to avoid confusion.
\end{itemize}