% =========================================
%              Chapter 1C.4
%      Diet and Cardiovascular Health
%          Created by Michael Tang
%               2025.02.13
% =========================================

\subsubsection{1C.4 Diet and Cardiovascular Health}
\paragraph{Weight Issues and CVD Risk}
\begin{itemize}
    \item \textbf{Obesity and CVDs:} Obesity, often caused by consuming more food than the body needs, increases the risk of
    cardiovascular diseases (CVDs) due to the \underline{accumulation} (积累) of \underline{excess} (过量的) fat.
    \item \textbf{BMI and Obesity:} The Body Mass Index (BMI) is commonly used to measure obesity. It is calculated as the
    following equation.
    \begin{equation}
        \text{BMI} = \frac{\text{Weight (kg)}}{\text{Height}^2 (\text{m}^2)}
    \end{equation}
\end{itemize}

\paragraph{Measuring Healthy Weight}
\begin{itemize}
    \item \textbf{BMI Classification for Adults:}
    \begin{itemize}
        \item Underweight: BMI $< 18.5 \text{kg}/\text{m}^2$
        \item Ideal weight: $18.5 \leq \text{BMI} < 25 \text{kg}/\text{m}^2$
        \item Overweight: $25 \leq \text{BMI} < 30 \text{kg}/\text{m}^2$
        \item Obese: BMI $\geq 30 \text{kg}/\text{m}^2$
        \item Severely obese: BMI $\geq 40 \text{kg}/\text{m}^2$
    \end{itemize}
    \item \textbf{\underline{Waist-to-Hip Ratio} (WHR 腰臀比):}
    \begin{itemize}
        \item \textbf{Waist size:} measured at the navel (肚脐) level.
        \item \textbf{Hip size:} measured at the widest part of the hips.
        \item \textbf{Indicator of Obesity:}
        \begin{itemize}
            \item \textbf{Male:} WHR $> 0.9$ indicates obesity.
            \item \textbf{Female:} WHR $> 0.85$ indicates obesity.
        \end{itemize}
        \item A high WHR indicates a higher risk of CVDs.
    \end{itemize}
\end{itemize}

\paragraph{BMI Limitations in Predicting CVD Risk}
\begin{itemize}
    \item \textbf{Muscle vs. Fat:} BMI doesn't account for muscle mass. Athletes or muscular individuals might be classified as
    overweight or obese even if they are healthy.
    \item \textbf{Genetic Variation:} People metabolize fats differently, and some can maintain a healthy balance of LDLs
    (Low-Density Lipoproteins 低密度脂蛋白) and HDLs (High-Density Lipoproteins 高密度脂蛋白) despite higher body fat.
\end{itemize}

\paragraph{\underline{Cholesterol} (胆固醇) and CVD Risk}
\begin{itemize}
    \item \textbf{LDL (Low-Density Lipoproteins):} Carry cholesterol from the liver to the cells. High LDL levels lead to plaque
    buildup in arteries, increasing the risk of CVD.
    \item \textbf{HDL (High-Density Lipoproteins):} Transport cholesterol from cells to the liver for disposal. Higher HDL levels
    reduce CVD risk.
    \item \textbf{Ideal LDL:HDL Ratio:} Around 3:1 (LDL:HDL).
\end{itemize}

\paragraph{Diet, Fat, and CVDs}
\begin{itemize}
    \item \textbf{Saturated Fats:} Linked to high cholesterol levels and increased risk of CVDs. Countries with high saturated
    fat intake (from animal fats) have higher heart disease rates.
    \item \textbf{Correlation between Fat Intake and CVDs:} Diets high in animal fats (mainly saturated fats) are associated with
    an increase in CVDs.
\end{itemize}
