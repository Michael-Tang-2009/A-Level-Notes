% ===============================================
%                 Chapter 1B.1
%  Techniques for Measuring the Rate of Reaction
%             Created by Michael Tang
%                  2024.12.31
% ===============================================

\subsubsection{1B.1 Writing Chemical Equations}
\paragraph{Writing Formulae for Names}
\begin{itemize}
    \item Familiarize yourself with the formulae of common compounds, and \underline{deduce} (推断) the formulae from given names.
    Examples:
    \begin{itemize}
        \item \textbf{Oxygen:} \ce{O2} not \ce{O}
        \item \textbf{Hydrogen:} \ce{H2} not \ce{H}
        \item \textbf{Nitrogen:} \ce{N2} not \ce{N}
        \item \textbf{Water:} \ce{H2O}
        \item \textbf{Sodium Hydroxide:} \ce{NaOH}
        \item \textbf{Nitric Acid:} \ce{HNO3}
    \end{itemize}
    \item Work out the formulae for:
    \begin{itemize}
        \item \textbf{Iron(II) Sulfate:} \ce{FeSO4}
        \item \textbf{Iron(III) Oxide:} \ce{Fe2O3}
        \item \textbf{Calcium Carbonate:} \ce{CaCO3}
    \end{itemize}
    \item \textbf{Exam Hint:} Use the periodic table to deduce formulae for compounds within the same group (e.g., Magnesium
    sulfate \ce{MgSO4} and strontium sulfate \ce{SrSO4}).
\end{itemize}

\paragraph{Balancing Equations}
\begin{itemize}
    \item Add \underline{coefficients} (系数) to ensure the number of atoms for each element is equal on both sides.
    \item Example:
    \begin{equation}
        \begin{split}
            \ce{H2O2 &-> H2O + O2} \\
            \text{Balanced Equation:} \quad \ce{2H2O2 &-> 2H2O + O2}
        \end{split}
    \end{equation}
\end{itemize}

\paragraph{State Stmbols:} Indicate the state of substances:
\begin{center}
    (s) = solid, (l) = liquid, (g) = gas, (aq) = aqueous (dissolved in water)
\end{center}

\paragraph{Ionic Equations}
\begin{itemize}
    \item \textbf{Simplifying Full Equations}
    \begin{itemize}
        \item Ionic equations include only the ions that participate in the reaction.
        \item Steps to simplify:
        \begin{itemize}
            \item[1.] Write the full balanced equation.
            \item[2.] Replace ionic compounds with their respective ions.
            \item[3.] Cancel out identical ions on both sides (spectator ions 旁观离子).
        \end{itemize}
    \end{itemize}
    \item \textbf{Worked Example 1:} \underline{Neutralization} (中和) of sodium hydroxide and nitric acid:\par
    The full balanced equation is:
    \begin{equation}
        \ce{NaOH(aq) + HNO3(aq) -> NaNO3(aq) + H2O(l)}
    \end{equation}
    You should now consider which of these species are ionic and replace them with ions. In this example, the first three
    compounds are ionic:
    \begin{equation}
        \ce{Na+(aq) + OH-(aq) + H+(aq) + NO3-(aq) -> Na+(aq) + NO3-(aq) + H2O(l)}
    \end{equation}
    After deleting the identical ions, the equation becomes:
    \begin{equation}
        \ce{OH-(aq) + H+(aq) -> H2O(l)}
    \end{equation}
    \item \textbf{Worked Example 2:} Reaction between lead(II) nitrate and sodium sulfate:\par
    The full balanced equation is:
    \begin{equation}
        \ce{Pb(NO3)2(aq) + Na2SO4(aq) -> PbSO4(s) + 2NaNO3(aq)}
    \end{equation}
    The ionic equation is:
    \begin{equation}
        \begin{split}
            \ce{Pb^{2+}(aq) + 2NO3-(aq) + 2Na+(aq) + SO4^{2-}(aq) \\ -> PbSO4(s) + 2Na+(aq) + 2NO3-(aq)}
        \end{split}
    \end{equation}
    After deleting the identical ions, the equation becomes:
    \begin{equation}
        \ce{Pb^{2+}(aq) + SO4^{2-}(aq) -> PbSO4(s)}
    \end{equation}
    \item \textbf{Worked Example 3:} Carbon dioxide reacts with calcium hydroxide:\par
    The full balanced equation is:
    \begin{equation}
        \ce{CO2(g) + Ca(OH)2(aq) -> CaCO3(s) + H2O(l)}
    \end{equation}
    The ionic equation is:
    \begin{equation}
        \ce{CO2(g) + Ca^{2+}(aq) + 2OH-(aq) -> CaCO3(s) + H2O(l)}
    \end{equation}
    After deleting the identical ions, the equation becomes:
    \begin{equation}
        \ce{CO2(g) + Ca^{2+}(aq) + 2OH-(aq) -> CaCO3(s) + H2O(l)}
    \end{equation}
\end{itemize}

\paragraph{Ionic \underline{Half-Equations} (半反应)}
\begin{itemize}
    \item Half-equations show the oxidation or reduction process of individual reactants.
    \item Example: Reduction during the electrolysis of sulfuric acid:
    \begin{equation}
        \begin{split}
            \ce{2H+(aq) + 2e- &-> H2(g)} \\
            \ce{2H2O(l) + 2e- &-> H2(g) + 2OH-(aq)}
        \end{split}
    \end{equation}
\end{itemize}