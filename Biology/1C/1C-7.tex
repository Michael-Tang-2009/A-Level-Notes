% =========================================
%              Chapter 1C.7
%    The Benefits and Risks of Treatment
%          Created by Michael Tang
%               2025.02.13
% =========================================

\subsubsection{1C.7 The Benefits and Risks of Treatment}
\paragraph{Controlling Blood Pressure (Antihypertensives 降压药)}
\begin{itemize}
    \item Hypertension (high blood pressure) is a major risk factor for CVDs.
    \item Antihypertensive drugs help lower blood pressure and prevent complications.
\end{itemize}

\paragraph{Types of Antihypertensive Drugs}
\begin{itemize}
    \item[1.] \textbf{\underline{Diuretics} (利尿剂):} Increase urine output, removing excess fluids and salts, reducing blood
    volume and lowering blood pressure.
    \item[2.] \textbf{\underline{Beta-blockers} ($\beta$-受体阻滞剂):} Reduce the effects of \underline{adrenaline} (肾上腺素),
    slowing the heart rate and making contractions less forceful.
    \item[3.] \textbf{\underline{Symphathetic Nervous Inhibitors} (交感神经系统抑制剂):} Prevent nerve signals that cause arteries
    to \underline{constrict} (收缩), keeping arteries relaxed.
    \item[4.] \textbf{\underline{ACE Inhibitors} (ACE 抑制剂):} Block the production of angiotensin (a hormone that constricts
    blood vessels 血管紧张素), keeping blood pressure low.
\end{itemize}

\paragraph{Risks of Antihypertensive Drugs}
\begin{itemize}
    \item If not monitored properly, blood pressure may drop too low, causing falls, dizziness, and \underline{fainting} (晕厥).
    \item \textbf{\underline{Side Effects} (副作用):} \underline{Swelling} (肿胀), tiredness, fatigue, \underline{constipation}
    (便秘).
    \item \textbf{\textbf{Compliance Issue} (服从性问题):} Patients may stop taking medication due to side effects, ignoring the 
    greater but invisible risk of CVDs.
\end{itemize}

\paragraph{Statins (Cholesterol-Lowering Drugs 他汀类药物 / 降胆固醇药物)}
\begin{itemize}
    \item Statins reduce blood cholesterol levels, particularly LDLs (bad cholesterol), while increasing HDLs (good cholesterol).
    \item \textbf{Function}
    \begin{itemize}
        \item Block the enzyme in the liver responsible for cholesterol production.
        \item Reduce \underline{inflammation} (发炎) in the arteries, preventing plaque buildup.
    \end{itemize}
    \item \textbf{Effectiveness}
    \begin{itemize}
        \item Statins significantly reduce the risk of CVDs across different groups (e.g., smokers, diabetics, elderly).
        \item Greatest effect seen in diabetics and smokers.
    \end{itemize}
    \item \textbf{Risks}
    \begin{itemize}
        \item Common side effects: Muscle/joint aches, \underline{nausea} (恶心), \underbar{constipation} (便秘), and
        \underline{diarrhea} (腹泻).
        \item Rare but serious risks:
        \begin{itemize}
            \item Muscle inflammation (can be fatal).
            \item Liver damage (very rare but possible).
        \end{itemize}
    \end{itemize}
    \item \textbf{Long-Term Effects (UK Study)}
    \begin{itemize}
        \item Over 5 years, statin users had a lower risk of heart attacks and death.
        \item Protection lasted up to 10 years after stopping the drug.
    \end{itemize}
\end{itemize}

\paragraph{\underline{Plant Stanols} (植物甾烷醇 / 植物固醇) and Sterols}
\begin{itemize}
    \item Found in spreads and yoghurts, similar in structure to cholesterol.
    \item Reduce LDL absorption in the blood, lowering heart disease risk by \~25\%.
    \item Not as rigorously tested as statins, but still beneficial in a healthy diet.
\end{itemize}

\paragraph{\underline{Anticoagulants} (抗凝血药) and Platelet Inhibitors (Preventing Blood Clots)}
\begin{itemize}
    \item Used after surgery or for people at risk of blood clots (thrombosis 血栓形成).
    \item Prevent blood clotting too easily, reducing heart attack and stroke risk.
    \item \textbf{Types}
    \begin{itemize}
        \item \underline{Anticoagulants} (抗凝血药, e.g., \underline{Warfarin} (华法林))
        \begin{itemize}
            \item Interferes with \underline{prothrombin} (凝血酶原 / 凝血素) production, preventing clot formation.
            \item \textbf{Risk:} If not monitored, internal bleeding can occur.
        \end{itemize}
        \item \underline{Platelet Inhibitory Drugs} (血小板抑制剂, e.g., \underline{Aspirin} (阿司匹林), \underline{Clopidogrel}
        (氯吡格雷))
        \begin{itemize}
            \item Prevent platelets from sticking together, reducing clot risk.
            \item \textbf{Risk:} Can cause stomach \underline{irritation} (刺激) and bleeding.
        \end{itemize}
    \end{itemize}
    \item \textbf{Balancing Risks and Benefits}
    \begin{itemize}
        \item Combination of aspirin + clopidogrel can reduce CVD risk by 20-25\%.
        \item However, it increases the risk of serious bleeding.
        \begin{itemize}
            \item \textbf{In high-risk patients:} 5 cardiovascular events avoided per 1000 treated, but 3 major bleeds occur.
            \item \textbf{In low-risk patients:} 23 cardiovascular events avoided, but 10 major bleeds occur.
        \end{itemize}
    \end{itemize}
\end{itemize}