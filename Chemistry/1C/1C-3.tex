% ===============================================
%                 Chapter 1C.3
%        Calculations Using Reacting Masses
%             Created by Michael Tang
%                  2025.01.04
% ===============================================

\subsubsection{1C.3 Calculations Using Reacting Masses}
\paragraph{Introduction to Reacting Masses}
\begin{itemize}
    \item A balanced chemical equation provides the relationship between the amounts of reactants and products.
    \item Example:
    \begin{equation}
        \ce{N2 + 3H2 -> 2NH3}
    \end{equation}
    \begin{itemize}
        \item 1 mole of \ce{N2} reacts with 3 moles of \ce{H2} to produce 2 moles of \ce{NH3}.
        \item In terms of mass:
        \begin{equation}
            28.0 \text{ g of \ce{N2}} + 6.0 \text{g of \ce{H2}} \ce{->} 34.0 \text{g of \ce{NH3}}
        \end{equation}
    \end{itemize}
\end{itemize}

\paragraph{Steps to Calculate Reacting Masses}
\begin{itemize}
    \item[1.] Write a balanced chemical equation.
    \item[2.] Calculate moles of the known substance using:
    \begin{equation}
        n = \frac{m}{M}
    \end{equation}
    \item[3.] Use the mole ratio from the balanced equation to calculate moles of the unknown substance.
    \item[4.] Calculate the mass of the other substance using:
    \begin{equation}
        m = n \times M
    \end{equation}
\end{itemize}