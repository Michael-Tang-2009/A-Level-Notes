% ===============================================
%                 Chapter 1C.4
%            The Yield of a Reaction
%             Created by Michael Tang
%                  2025.01.04
% ===============================================

\subsubsection{1C.4 The Yield of a Reaction}
\paragraph{\underline{Theoretical Yield} (理论产量), \underline{Actual Yield} (实际产量), and \underline{Percentage Yield} (产率)}
\paragraph{Introduction}
\begin{itemize}
    \item In laboratory and industrial chemistry, it is essential to maximize product yield.
    \item Factors reducing yield:
    \begin{itemize}
        \item The reaction may be \underline{reversible} (可逆的) and not go to completion.
        \item Side-reactions producing unwanted products.
        \item \underline{Purification} (提纯) steps leading to product loss.
    \end{itemize}
\end{itemize}

\paragraph{Terminology Relating to Yield}
\begin{itemize}
    \item[1.] \textbf{Theoretical Yield:}
    \begin{itemize}
        \item The maximum possible amount of product calculated from the balanced chemical equation.
        \item \underline{Assumes} (假设) the reaction goes to completion with no losses.
    \end{itemize}
    \item[2.] \textbf{Actual Yield:}
    \begin{itemize}
        \item The actual mass of product \underline{obtained} (获得) from an experiment.
        \item Measured by weighing the final product.
    \end{itemize}
    \item[3.] \textbf{Percentage Yield:}
    \begin{itemize}
        \item A measure of the efficiency of the reaction:
        \begin{equation}
            \text{Percentage Yield} = \frac{\text{Actual Yield}}{\text{Theoretical Yield}} \times 100\%
        \end{equation}
        \item Indicates how much of the theoretical yield was obtained.
        \item Higher yields indicate more efficient processes.
    \end{itemize}
\end{itemize}

\paragraph{Applications in Industry}
\begin{itemize}
    \item Percentage yield is crucial for cost efficiency.
    \item High yields reduce waste and maximize resource utilization.
\end{itemize}