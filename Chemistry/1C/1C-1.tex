% ===============================================
%                 Chapter 1C.1
%         Comparing Masses of Substances
%             Created by Michael Tang
%                  2025.01.04
% ===============================================

\subsubsection{1C.1 Comparing Masses of Substances}
\paragraph{Relative Atomic Mass ($A_r$ 相对原子质量)}
\begin{itemize}
    \item \textbf{Definition:} The weighted mean (average) mass of an atom of an elemet compared to $\frac{1}{12}$ of the mass of
    an atom of \ce{^{12}C} (carbon-12).
    \begin{equation}
        A_r = \frac{\text{Mean mass of an atom of the element}}{\frac{1}{12}\text{ of the mass of an atom of \ce{^{12}C}}}
    \end{equation}
    \item \textbf{Key Points:}
    \begin{itemize}
        \item Based on isotopic masses.
        \item Has no units.
        \item Found in the preiodic table.
    \end{itemize}
\end{itemize}

\paragraph{Relative Molecular Mass ($M_r$ 相对分子质量)}
\begin{itemize}
    \item \textbf{Definition:} Sum of the relative atomic masses ($A_r$) of the atoms in a molecule.
    \item \textbf{Examples:}
    \begin{itemize}
        \item \ce{CO2}:
        \begin{equation}
            M_r = 12.0 + (2 \times 16.0) = 44.0
        \end{equation}
        \item \ce{H2SO4}:
        \begin{equation}
            M_r = (2 \times 1.0) + 32.1 + (4 \times 16.0) = 98.1
        \end{equation}
    \end{itemize}
\end{itemize}

\paragraph{Relative Formula Mass ($M_r$ 相对式量/化学式质量)}
\begin{itemize}
    \item \textbf{Definition:} Silimar to $M_r$, but used for ionic compounds (e.g., \ce{NaCl}, \ce{CuSO4} $\cdot$ \ce{5H2O}).
    \item \textbf{Example:} \ce{CuSO4} $\cdot$ \ce{5H2O}:
    \begin{equation}
        M_r = 63.5 + 32.1 + (4 \times 16.0) + (5 \times 18.0) = 249.6
    \end{equation}
\end{itemize}

\paragraph{Molar Mass ($M$ 摩尔质量)}
\begin{itemize}
    \item \textbf{Definition:} The mass of one mole of a substance; has units \unit{g.mol^{-1}}.
    \item \textbf{Formula:}
    \begin{equation}
        \begin{split}
            \text{Amount in moles} &= \frac{\text{Mass of substance (} \unit{g} \text{)}}{\text{Molar mass (} \unit{g.mol^{-1}}
            \text{)}}\\
            \text{or} \quad n &= \frac{m}{M}
        \end{split}
    \end{equation}
    \item \textbf{Examples:}
    \begin{itemize}
        \item \ce{O2}: Molar mass = 32.0 \unit{g.mol^{-1}}
        \item \ce{H2O}: Molar mass = 18.0 \unit{g.mol^{-1}}
    \end{itemize}
\end{itemize}

\paragraph{The Avogadro Constant ($N_A$ 阿伏伽德罗常数)}
\begin{itemize}
    \item \textbf{Definition:} The number of particles (atoms, molecules, or ions) in one mole of a substance.
    \begin{equation}
        N_A = 6.02 \times 10^{23} \unit{mol^{-1}}
    \end{equation}
    \item \textbf{Applications:}
    \begin{itemize}
        \item Number of particles:
        \begin{equation}
            \text{Number of particles} = \text{Amount in moles} \times N_A
        \end{equation}
        \item Example:
        \begin{equation}
            \begin{split}
                n = \frac{1.25 \unit{g} \text{ of } \ce{H2O}}{18.0 \unit{g.mol^{-1}}} = 0.0694 \unit{mol}\\
                \text{Number of molecules} = 0.0694 \times 6.02 \times 10^{23} = 4.18 \times 10^{22}
            \end{split}
        \end{equation}
    \end{itemize}
\end{itemize}

\paragraph{Worked Examples}
\begin{itemize}
    \item[1.] Calculate $M_r$ of \ce{H2SO4}:
    \begin{equation}
        M_r = (2 \times 1.0) + 32.1 + (4 \times 16.0) = 98.1
    \end{equation}
    \item[2.] Number of Particles in 1.25 \unit{g} of \ce{H2O}:
    \begin{equation}
        \begin{split}
            n &= \frac{1.25 \unit{g} \text{ of } \ce{H2O}}{18.0 \unit{g.mol^{-1}}} = 0.0694 \unit{mol}\\
            \text{Number of molecules} &= 0.0694 \times 6.02 \times 10^{23} = 4.18 \times 10^{22}
        \end{split}
    \end{equation}
    \item[3.] Mass of 100 Million Atoms of Gold:
    \begin{equation}
        \begin{split}
            n &= \frac{100 \times 10^6 \text{ atoms}}{6.02 \times 10^{23} \unit{mol^{-1}}} = 1.66 \times 10^{-16} \unit{mol}\\
            m &= 1.66 \times 10^{-16} \times 197.0 = 3.27 \times 10^{-14} \unit{g}
        \end{split}
    \end{equation}
\end{itemize}