% ===============================
%          Chapter 1.6
%   Rationalising Denominators
%     Created by Michael Tang
%           2025.02.23
% ===============================

\subsection{Rationalising Denominators}
\begin{itemize}
    \item Rationalising the denominator is a technique used to remove surds from the denominator of a fraction.
    \item \textbf{Rules to rationalise denominators}
    \begin{itemize}
        \item[1.] For fractions of the form $\frac{1}{\sqrt{a}}$, multiply both the numerator and denominator by $\sqrt{a}$.
        \begin{equation}
            \frac{1}{\sqrt{a}} = \frac{1}{\sqrt{a}} \times \frac{\sqrt{a}}{\sqrt{a}} = \frac{\sqrt{a}}{a}
        \end{equation}
        \item[2.] For fractions of the form $\frac{1}{a + \sqrt{b}}$, multiply both the numerator and denominator by the
        \underline{conjugate} \footnote{In mathematics, conjugate refers to a pair of expressions or numbers that are related in
        a specific way. For example:
        \begin{itemize}
            \item In complex numbers, the conjugate of a complex number $a + bi$ is $a - bi$, where $a$ and $b$ are real numbers
            and $i$ is the \underline{imaginary unit} \footnotemark (虚数单位). The conjugates of complex numbers have useful
            properties, especially when simplifying expressions or dividing complex numbers. For instance, multiplying a complex
            number by its conjugate results in a real number, as the imaginary parts cancel out.
            \item In algebra, when dealing with binomials, the conjugate of a binomial like $\left(a + b\right)$ is
            $\left(a - b\right)$. Conjugates are often used to simplify expressions, particularly when rationalizing denominators
            or working with square roots.
        \end{itemize}
        The conjugate has many applications in solving equations and simplifying expressions in algebra, calculus, and complex
        number theory.} (共轭) of the denominator by $a - \sqrt{b}$.
        \begin{equation}
            \frac{1}{a + \sqrt{b}} = \frac{1}{a + \sqrt{b}} \times \frac{a - \sqrt{b}}{a - \sqrt{b}} = \frac{a - \sqrt{b}}{a^2 - b}
        \end{equation}
        \item[3.] For fractions of the form $\frac{1}{a - \sqrt{b}}$, multiply both the numerator and denominator by
        $a + \sqrt{b}$.
        \begin{equation}
            \frac{1}{a - \sqrt{b}} = \frac{1}{a - \sqrt{b}} \times \frac{a + \sqrt{b}}{a + \sqrt{b}} = \frac{a + \sqrt{b}}{a^2 - b}
        \end{equation}
    \end{itemize}
\end{itemize}

\footnotetext[2]{The imaginary unit, denoted as $i$, is a mathematical concept used to define complex numbers. It is defined as
the square root of $-1$:
\begin{equation}
    i = \sqrt{-1}
\end{equation}
This definition is fundamental because no real number has a square root that results in a negative value. Therefore, the imaginary
unit allows for the extension of real numbers to complex numbers.\par
A complex number is a number that has both a real part and an imaginary part, and is generally written in the form:
\begin{equation}
    a + bi
\end{equation}
where $a$ and $b$ are real numbers, and $i$ is the imaginary unit.\par
Some important properties of the imaginary unit include:
\begin{itemize}
    \item $i^2 = -1$
    \item $i^3 = -i$
    \item $i^4 = 1$
    \item $i^5 = i$
    \item $i^6 = -1$
\end{itemize}
These properties repeat in a cycle, which is crucial in many algebraic manipulations, especially when working with powers of
complex numbers. The imaginary unit is central to complex analysis, engineering, and physics.}