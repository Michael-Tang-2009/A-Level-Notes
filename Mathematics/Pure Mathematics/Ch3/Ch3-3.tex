% ==================================
%            Chapter 3.3
%  Simultaneous Equations on Graphs
%       Created by Michael Tang
%             2025.02.25
% ==================================

\subsection{Simultaneous Equations on Graphs}
\begin{itemize}
    \item \textbf{Key Concepts}
    \begin{itemize}
        \item The solutions to simultaneous equations represent the points where the graphs of the equations intersect.
        \item We can graph linear equations, quadratic equations, or a combination of both to find the points of intersection,
        which correspond to the solutions of the system of equations.
    \end{itemize}
    \item \textbf{Types of Intersections}
    \begin{itemize}
        \item[1.] \textbf{Linear vs Linear:} The graphs of two linear equations intersect at one point unless they are parallel
        (no solution).
        \item[2.] \textbf{Linear vs Quadratic:} A linear equation and a quadratic equation can intersect at one or two points. The
        intersection points are the solutions to the system.
        \item[3.] \textbf{Quadratic vs Quadratic:} A quadratic equation and another quadratic equation may intersect at two
        points, one point, or not intersect at all.
    \end{itemize}
    \item \textbf{Form of Equations}
    \begin{itemize}
        \item \textbf{Linear Equation:}
        \begin{itemize}
            \item \textbf{General Formula}
            \begin{equation}
                y = kx + b
            \end{equation}
            \item Where $k$ is the slope (gradient) and $b$ is the y-intercept.
        \end{itemize}
        \item \textbf{Quadratic Equation:}
        \begin{itemize}
            \item \textbf{General Formula}
            \begin{equation}
                y = ax^2 + bx + c
            \end{equation}
            \item Where $a$ is the \underline{coefficient} (系数) that affects the \underline{curvature} (曲率) of the graph, $b$
            is the coefficient of $x$, and $c$ is the y-intercept.
        \end{itemize}
    \end{itemize}
    \item \textbf{Graphing Simultaneous Equations}
    \begin{itemize}
        \item \textbf{Intersection of two lines:} Solve algebraically by substitution or elimination.
        \item \textbf{Intersection of a line and a curve:} Use substitution to substitute the linear equation into the quadratic
        one, and solve for the variable(s).
        \item \textbf{Intersection of two curves:} Set up both equations and solve the resulting system algebraically.
    \end{itemize}
    \item \textbf{Discriminant and Number of Solutions} The number of intersection points between two equations, particularly when
    involving quadratics, can be analyzed by examining the discriminant:
    \begin{equation}
        \Delta = b^2 - 4ac
    \end{equation}
    \begin{itemize}
        \item If $\Delta > 0$, there are two real solution (two points of intersection).
        \item If $\Delta = 0$, there is exactly one solution (one point of intersection).
        \item If $\Delta < 0$, there are no real solutions (no intersection).
    \end{itemize}
\end{itemize}

\paragraph{Graph Interpretation and Problem Solving}
\begin{itemize}
    \item \textbf{Graphing Linear Equations:} Plot lines carefully, check their slope and intercept, and identify intersection
    points by solving algebraically.
    \item \textbf{Graphing Quadratic Equations:} Understand the shape of the parabola (upward or downward) and its vertex.
    \item \textbf{Intersecting Curves:} When a linear equation intersects a quadratic, the point(s) of intersection provide the
    solution(s) to the system.
\end{itemize}