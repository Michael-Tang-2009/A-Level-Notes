% =================================================
%                  Chapter 1C.5
%  Dietary Antioxidants and Cardiovascular Disease
%             Created by Michael Tang
%                   2025.02.13
% =================================================

\subsubsection{1C.5 Dietary Antioxidants and Cardiovascular Disease}
\paragraph{\underline{Antioxidants} (防老化) and Heart Health}
\begin{itemize}
    \item \textbf{Antioxidants:} Compounds found in fruits and vegetables that may reduce oxidative stress, which can contribute
    to CVDs.
    \item \textbf{Fruit and Vegetables:} Eating 5 or more portions of fruits and vegetables per day is linked to a lower risk of
    coronary heart disease (CVD), as seen in data from a longitudinal study.
    \item \textbf{Vitamin C:} An antioxidant known for its role in connective tissue and blood vessel health. It helps protect
    against damage in the arteries. A deficiency in vitamin C can lead to a higher risk of heart disease, as the body's blood
    vessels are more prone to damage.
\end{itemize}

\paragraph{Conflicting Evidence in Antioxidant Research}
\begin{itemize}
    \item Early studies suggested antioxidants like vitamin C could prevent heart disease.
    \item \textbf{Conflicting Evidence:} A major study in 2016 concluded that there was no clear relationship between vitamin C
    intake and heart health. It even showed that taking vitamin C supplements might damage heart health.
\end{itemize}

\paragraph{Testing for Vitamin C}
\begin{itemize}
    \item A simple laboratory test can be used to measure vitamin C in foods. The test uses a reagent called \underline{DCPIP}
    (2,6-Dichlorophenolindophenol 二氯酚靛酚, \ce{C12H7Cl2NO2}), which turns colorless when it reacts with vitamin C.
    \item The volume of DCPIP used indicates the concentration of vitamin C in the sample.
    \item \textbf{Application:} The test can be used to compare the vitamin C content in different foods
\end{itemize}