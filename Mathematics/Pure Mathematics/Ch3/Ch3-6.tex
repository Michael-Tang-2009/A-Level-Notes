% ==================================
%            Chapter 3.6
%       Inequalities on Graphs
%       Created by Michael Tang
%             2025.02.25
% ==================================

\subsection{Inequalities on Graphs}
\begin{itemize}
    \item \textbf{Key Concept:} Inequalities on graphs can be interpreted by comparing the graphs of two functions. The inequality
    depends on which function lies above or below the other function.
    \item \textbf{Graph Interpretation for Inequalities}
    \begin{itemize}
        \item[1.] For $y > f(x)$ or $y < f(x)$:
        \begin{itemize}
            \item The region where $y > f(x)$ represents the area above the graph of $y = f(x)$.
            \item The region where $y < f(x)$ represents the area below the graph of $y = f(x)$.
            \item When the inequality involves $>$ or $<$, the graph of $y = f(x)$ is not included in the region, and the line is
            represented as a dotted line.
        \end{itemize}
        \item[2.] For $y \geq f(x)$ or $y \leq f(x)$: When the inequality involves $\geq$ or $\leq$, the graph of $y = f(x)$ is
        included in the region, and the line is represented as a solid line.
    \end{itemize}
\end{itemize}