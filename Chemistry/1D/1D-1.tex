% ===============================================
%                 Chapter 1D.1
%               Empirical Formulae
%             Created by Michael Tang
%                  2025.01.05
% ===============================================

\subsubsection{1D.1 Empirical Formulae}
\paragraph{The Definition of \underline{Empirical Formulae} (实验式)}
\begin{itemize}
    \item An empirical formula represents the simplest whole-number ratio of atoms of each element in a compound.
    \item It is determined from experimental data such as masses or percentage compositions.
\end{itemize}

\paragraph{Steps to Calculate Empirical Formula}
\begin{itemize}
    \item[1.] Divide the mass or percentage composition of each element by its relative atomic mass ($A_r$).
    \item[2.] Calculate the ratio of the elements by dividing all results by the smallest number obtained.
    \item[3.] If necessary, round to the nearest whole number or multiply to achieve whole numbers
\end{itemize}

\paragraph{Example Using Masses}
Determining the formula of copper oxide (\ce{CuO}):
\begin{itemize}
    \item Mass of copper: 3.43 g, mass of oxygen: 0.85 g
    \item Relative atomic masses: $\ce{Cu} = 63.5$, $\ce{O} = 16.0$.
    \item Steps:
    \begin{itemize}
        \item[1.] Divide masses by $A_r$:
        \begin{equation}
            \frac{3.43}{63.5} = 0.0540, \quad \frac{0.85}{16.0} = 0.0531
        \end{equation}
        \item[2.] Simplify ratio
        \begin{equation}
            \frac{0.0540}{0.0531} \approx 1:1
        \end{equation}
        \item[3.] Empirical formula: \ce{CuO}.
    \end{itemize}
\end{itemize}

\paragraph{Example Using Percentage Composition}
Compound with $\ce{C} = 38.4 \%, \quad \ce{H} = 4.8 \%, \quad \ce{Cl} = 56.8 \%$:
\begin{itemize}
    \item Relative atomic masses: $\ce{C} = 12.0$, $\ce{H} = 1.0$, $\ce{Cl} = 35.5$.
    \item Steps:
    \begin{itemize}
        \item[1.] Divide percentages by $A_r$:
        \begin{equation}
            \frac{38.4}{12.0} = 3.2, \quad \frac{4.8}{1.0} = 4.8, \quad \frac{56.8}{35.5} = 1.6
        \end{equation}
        \item[2.] Simplify ratio:
        \begin{equation}
            3.2 : 4.8 : 1.6 \approx 2 : 3 : 1
        \end{equation}
        \item[3.] Empirical formula: $\ce{C2H3Cl}$.
    \end{itemize}
\end{itemize}

\paragraph{Handling Oxygen as a Missing Value}
Example: $\ce{Na} = 29.1 \%, \quad \ce{S} = 40.5 \%$, oxygen not provided.
\begin{itemize}
    \item Calculate oxygen by subtraction:
    \begin{equation}
        \text{Oxygen Percentage} = 100 - \left(29.1 + 40.5\right) = 30.4 \%
    \end{equation}
    \item Determine ratio:
    \begin{equation}
        \frac{29.1}{23.0} = 1.27, \quad \frac{40.5}{32.1} = 1.26, \quad \frac{30.4}{16.0} = 1.90
    \end{equation}
    \item Simplify ratio:
    \begin{equation}
        1.27 : 1.26 : 1.90 \approx 1 : 1 : 2
    \end{equation}
    \item Empirical formula: $\ce{Na2S2O3}$.
\end{itemize}

\paragraph{Using \underline{Combustion Analysis} (燃烧分析)}
Determine the mass of \ce{C}, \ce{H}, and \ce{O} from the masses of \ce{CO2} and \ce{H2O} produced.
\begin{equation}
    \begin{split}
        \ce{C} &= \frac{\text{Mass of } \ce{CO2}}{44.0} \times 12.0 \\
        \ce{H} &= \frac{\text{Mass of } \ce{H2O}}{18.0} \times 2.0 \\
        \ce{O} &= \text{Total mass} - \left(\ce{C} + \ce{H}\right)
    \end{split}
\end{equation}