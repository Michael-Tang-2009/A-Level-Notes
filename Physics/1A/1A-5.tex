% ===========================
%        Chapter 1A.5
%    Newton's Laws of Motion
%   Created by Michael Tang
%         2025.02.28
% ===========================

\subsubsection{1A.5 Newton's Laws of Motion}
\paragraph{Key Definitions}
\begin{itemize}
    \item \textbf{Newton's First Law of Motion:} An object will remain at rest, or continue at constant velocity in a straight
    line, unless acted upon by a resultant external force. This is the principle of \underline{inertia} (惯性) - no change in
    motion without an unbalanced force.
    \item \textbf{Newton's Second Law of Motion:} The acceleration of an object is directly proportional to the ner force acting
    on it and inversely proportional to its mass. In formula form:
    \begin{equation*}
        \begin{split}
            \text{Force} &= \text{Mass} \times \text{Acceleration} \\
            F &= ma \\
            &= mg \\
            \therefore \quad g &= \frac{F}{m}
        \end{split}
    \end{equation*}
    A resultant force $F$ (in newtons, \unit{N}) acting on mass $m$ (in kilograms, \unit{kg}) produces acceleration $a$ (in
    meters per second squared, \unit{m/s^2}) in the same direction.
    \item \textbf{Newton's Third Law of Motion:} For every action force, there is an equal and opposite reaction force. This
    means that if object A \underline{exerts} (施加) a force on object B, then object B \underline{simultaneously} (同时) exerts
    an equal force in the opposite direction on object A. These two forces are an interaction pair and of the same type.
\end{itemize}

\paragraph{Theoretical Concepts}
\begin{itemize}
    \item \textbf{Inertia and Newton's First Law:} Newton's first law defines inertia: an object's tendency to resist changes in
    motion. If no resultant force acts on an object, its velocity remains constant (if stationary, it stays at rest; if moving,
    it continues at the same speed in the same direction). In practical items, a book on a table will remain stationary until a
    net force (e.g. a push) is applied.
    \item \textbf{Newton's Second Law and Acceleration:} Newton's second law quantifies how forces affect motion. It says
    acceleration $a$ is proportional to force $F$ and inversely proportional to mass $m$. For a given mass, a larger force
    produces a larger acceleration; for a give force, a larger mass yields a smaller acceleration. \par
    For example, pushing two carts with the same force, the lighter cart accelerates more. This law also \underline{implies}
    (表明) the direction of the acceleration is the same as the direction of the net force. If multiple forces act, we sum them
    \underline{vectorially} (矢量地) to get the resultant force $\Sigma F$ and then apply $F = ma$. (If $\Sigma F = 0$, then
    $a = 0$, consistent with the first law.)
    \item \textbf{Newton's Third Law and Interaction Pairs:} Newton's third law emphasizes that forces always come in pairs.
    These \underline{acrion-reaction pairs} (相互作用对) act on different objects but are equal in magnitude and opposite in
    direction. \par
    For instance, consider a skateboarder pushing against a wall: the skateboarder exerts a force on the wall, and the wall
    exerts an equal and opposite force back on the skateboarder. Only because these forces act on different bodies can both
    objects accelerate (the skateboarder lunches backward while the wall remains essentially unmoved due to its large mass). \par
    It's crucial to note the two forces in a third-law pair do not cancel out because they act on different systems. The third
    law is often \underline{demonstrated} (证实) by the example of a kicking a football: the foot exerts a force on the ball, and
    the ball exerts an equal and opposite force on the foot. This reaction force is what causes the \underline{sensation} (感觉)
    of the kick.
    \item \textbf{Equilibrium:} When all forces on an object balance out (net force = 0), the object is in equilibrium. According
    to Newton's first law, it will either remain at rest or move with constant velocity. For example, a book resting on a table
    is in equilibrium: the upward normal force from the table equals the book's weight (downward force of gravity), resulting in
    no acceleration.
\end{itemize}