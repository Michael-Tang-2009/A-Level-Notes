% ===============================
%          Chapter 2.4
%        Quadratic Graphs
%     Created by Michael Tang
%           2025.02.24
% ===============================

\subsection{Quadratic Graphs}
\begin{itemize}
    \item \textbf{Key Features of Quadratic Graphs}
    \begin{itemize}
        \item The graph of $f(x) = ax^2 + bx + c$ is a \underline{parabola} (抛物线).
        \item \textbf{Shape of the graph}
        \begin{itemize}
            \item If $a > 0$, the parabola opens upwards (minimum point).
            \item If $a < 0$, the parabola opens downwards (maximum point).
        \end{itemize}
        \item \textbf{\underline{Vertex} (顶点):} The vertex of the parabola is the turning point of the graph. This can be found
        by completing the square or using the formula:
        \begin{equation}
            x_{\text{vertex}} = -\frac{b}{2a}
        \end{equation}
        For the function $f(x) = ax^2 + bx + c$, the x-coordinate of the vertex is $-\frac{b}{2a}$.
    \end{itemize}
    \item \textbf{Roots of the Graph}
    \begin{itemize}
        \item The roots (or solutions) of the quadratic equation are where the graph intersects the x-axis, i.e., $f(x) = 0$.
        \item The number of real roots is determined by the \underline{discriminant} (判别式) $\Delta = b^2 - 4ac$:
        \begin{itemize}
            \item If $\Delta > 0$, the graph has two distinct real roots.
            \item If $\Delta = 0$, the graph has one repeated real root.
            \item If $\Delta < 0$, the graph has no real roots (the graph does not intersect the x-axis).
        \end{itemize}
    \end{itemize}
\end{itemize}