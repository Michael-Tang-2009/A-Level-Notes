% ===============================================
%                 Chapter 1C.5
%                 Atom Economy
%             Created by Michael Tang
%                  2025.01.04
% ===============================================

\subsubsection{1C.5 Atom Economy}
\paragraph{Background}
\begin{itemize}
    \item \textbf{Definition:} Atom economy measures the efficiency of a reaction in converting reactants into useful products.
    \item \textbf{Importance:}
    \begin{itemize}
        \item A higher atom economy reduces waste and increases the \underline{sustainability} (持续性) of industrial processes.
        \item Factors influencing industrial suitability:
        \begin{itemize}
            \item Availability and cost of raw materials.
            \item Energy requirements.
            \item Environmental impact of waste products.
        \end{itemize}
    \end{itemize}
\end{itemize}

\paragraph{How Atom Economy Works}
Example: Manufacture of phosphoric acid (\ce{H3PO4}):
\begin{itemize}
    \item Process 1
    \begin{equation}
        \ce{Ca3(PO4)2 + 3H2SO4 -> 2H3PO4 + 3CaSO4}
    \end{equation}
    Many atoms end up in the waste product, calcium sulfate (\ce{CaSO4}). The atom economy is lower.
    \item Process 2
    \begin{equation}
        \ce{P4O10 + 6H2O -> 4H3PO4}
    \end{equation}
    All atoms in reactants form the desired product. Higher atom economy.
\end{itemize}

\paragraph{Formula for Atom Economy}
\begin{equation}
    \text{Atom Economy} = \frac{\text{Molar Mass of Desired Product}}{\text{Sum of Molar Masses of All Products}} \times 100\%
\end{equation}
Developed by Barry Trost to evaluate reaction efficiency in industrial processes.
\begin{itemize}
    \item Worked Example for Process 1:
    \begin{itemize}
        \item Desired product: \ce{2H3PO4}.
        \begin{itemize}
            \item Molar mass of \ce{H3PO4} $= 98.0$ g/mol.
            \item Total mass of product $= \left(98.0 \times 2\right) + \left(136.2 \times 3\right) = 644.6$ g.
        \end{itemize}
        \item Atom economy:
        \begin{equation}
            \frac{\left(98.0 \times 2\right)}{\left(98.0 \times 2\right) + \left(136.2 \times 3\right)} \times 100\% = 32.4\%
        \end{equation}
    \end{itemize}
\end{itemize}

\paragraph{Reaction Types and Atom Economy}
\begin{itemize}
    \item Addition reactions have 100\% atom economy.
    \item Elimination and substitution reactions have lower atom economies.
    \item Multistep reactions may have even lower atom economies.
\end{itemize}