% =========================================
%              Chapter 1C.3
%  Risk Factors for Cardiovascular Disease
%          Created by Michael Tang
%               2025.01.13
% =========================================

\subsubsection{1C.3 Risk Factors for Cardiovascular Disease}
\paragraph{Cardiovascular Disease Overview}
\begin{itemize}
    \item \textbf{Definition:} Diseases affecting the heart and blood vessels.
    \item \textbf{Global Impact:}
    \begin{itemize}
        \item Account for 31\% of global deaths (WHO data, 2017).
        \item Significant causes include atherosclerosis.
    \end{itemize}
\end{itemize}

\paragraph{Atherosclerosis}
\begin{itemize}
    \item \textbf{Process:}
    \begin{itemize}
        \item Damage to \underline{endothelium} (内皮细胞) due to high pressure or smoking.
        \item \underline{Inflammatory} (炎症性的) response $\rightarrow$ Cholesterol accumulates forming an atheroma.
        \item Atheromas harden into plaques causing artery narrowing and reduced \underline{elasticity} (弹性).
    \end{itemize}
    \item \textbf{Effects:}
    \begin{itemize}
        \item Increased blood pressure.
        \item Risks of \underline{aneurysms} (颅内动脉瘤), \underline{angina} (心绞痛), \underline{myocardial infarction} (心肌梗死),
        and stroke.
    \end{itemize}
\end{itemize}

\paragraph{Heart Disease}
\begin{itemize}
    \item \textbf{Angina:}
    \begin{itemize}
        \item Chest pain from restricted blood flow to the heart.
        \item Triggered by \underline{anaerobic respiration} \footnote{\textbf{Anaerobic Respiration} Anaerobic respiration is the
        breakdown of glucose without oxygen to release energy. It occurs in the \underline{cytoplasm} (细胞质) of cells and
        produces less ATP compared to aerobic respiration.
        \textbf{Process:}
        \begin{itemize}
            \item \textbf{In Animals:}
            \begin{itemize}
                \item Glucose $\rightarrow$ \underline{Lactic Acid} (乳酸) + Energy (ATP).
                \item Reaction Formula:
                \begin{equation}
                    \ce{C6H12O6 -> 2C3H6O3 + ATP}
                \end{equation}
                \item Key Point: Lactic acid builds up and can cause muscle \underline{fatigue} (劳累) during
                \underline{strenuous} (费力的) activities.
            \end{itemize}
            \item \textbf{In Plants and \underline{Yeast} (酵母):}
            \begin{itemize}
                \item Glucose $\rightarrow$ \underline{Ethanol} (乙醇) + Carbon Dioxide + Energy (ATP).
                \item Reaction Formula:
                \begin{equation}
                    \ce{C6H12O6 -> 2C2H5OH + 2CO2 + ATP}
                \end{equation}
                \item Key Point: Ethanol is produced and used in the production of alcoholic beverages.
            \end{itemize}
        \end{itemize}} (无氧呼吸) in \underline{cardiac muscles} (心肌).
    \end{itemize}
    \item \textbf{Myocardial Infarction (Heart Attack):}
    \begin{itemize}
        \item Complete blockage of \underline{coronary arteries} (冠状动脉) $\rightarrow$ Heart muscle is
        \underline{oxygen-deprived} (氧气供应不足).
        \item Severe cases lead to \underline{cardiac arrest} (心脏骤停) and death.
    \end{itemize}
\end{itemize}

\paragraph{Stroke}
\begin{itemize}
    \item \textbf{Cause:} Interruption of blood supply to the brain (clot or \underline{rupture} 破裂).
    \item \textbf{Symptoms:} Dizziness, speech issues, \underline{numbness} (麻木) on one side.
    \item \textbf{Survival:} Quick medical intervention significantly increases recovery chances.
\end{itemize}

\paragraph{Risk Factors}
\begin{itemize}
    \item \textbf{Non-Modifiable Risk Factors:}
    \begin{itemize}
        \item \textbf{Age:} Increased risk as blood vessels lose elasticity.
        \item \textbf{Gender:}
        \begin{itemize}
            \item Males at higher risk.
            \item Females post-menopause are at increased risk due to reduced \underline{oestrogen} (雌性激素) levels.
        \end{itemize}
        \item \textbf{Genetics:} Family history of CVD or inheritable artery weakness.
    \end{itemize}
    \item \textbf{Modifiable Risk Factors:}
    \begin{itemize}
        \item \textbf{Smoking:}
        \begin{itemize}
            \item Increases plaque formation.
            \item Damages \underline{endothelium} (内皮细胞) and narrows arteries.
        \end{itemize}
    \end{itemize}
\end{itemize}